\documentclass[a4paper, 12pt]{article}
\usepackage[a4paper,top=1.5cm, bottom=1.5cm, left=1cm, right=1cm]{geometry}
\usepackage{cmap}					% поиск в PDF
\usepackage{mathtext} 				% русские буквы в формулах
\usepackage[T2A]{fontenc}			% кодировка
\usepackage[utf8]{inputenc}			% кодировка исходного текста
\usepackage[english,russian]{babel}	% локализация и переносы

\usepackage{amsmath}
\usepackage{indentfirst}
\usepackage{longtable}
\usepackage{graphicx}
\usepackage{array}

\usepackage{wrapfig}
\usepackage{siunitx} % Required for alignment
\usepackage{subfigure}
\usepackage{multirow}
\usepackage{rotating}
\usepackage{caption}

\graphicspath{{.}}


\title{\begin{center} Название работы\end{center}
Измерение коэффициента поверхностного натяжения жидкости}
\author{Владимир Вехов // ФРКТ}
\date{\today}

\begin{document}
    \pagenumbering{gobble}
    \maketitle
    \newpage
    \pagenumbering{arabic}

    \textbf{Цель работы:} 1)...

	\textbf{В работе используются:} ...

    \section{Теоретические сведения}

	\begin{equation}
		\Delta P = P_{int} - P_{ext} = \frac{2\sigma}{r},
		\label{key}
	\end{equation}
	где $ \sigma $ -- коэффициент, $ P_{int} $ и $ P_{ext} $ -- давление

	\section{Экспериментальная установка}

    \begin{figure}[!ht]
        \begin{center}
            \includegraphics[width=0.6 \textwidth]{ustan.jpg}
        \end{center}
        \caption{Экспериментальная установка}
		\label{img:ust}
    \end{figure}


	\section{Ход работы}

		\subsection{Проверка герметичности установки}

			Для проверки опускаем чистую сухую иглу в сосуд со спиртом так, чтобы кончик иглы лишь касался поверхности спирта. Плотно закрываем обе колбы и открываем кран аспиратора для пробулькивания пузырьков воздуха в колбе. Замеряем показания микроманометра, они не должны меняться.

		\subsection{Начало измерений}

			Положением крана аспиратора подбираем частоту падения капель так, чтобы максимальное давление манометра не зависело от этой частоты (не чаще чем 1 капля в 5 секунд).

		\subsection{Определение диаметра иглы}
			\label{needle_diam}

			Параметры при измерениях:

			\begin{align*}
				t_{комн} &= 25~^oC & \sigma_{спирт} &= 22.3~\frac{мН}{м}
			\end{align*}


			\begin{table}[!ht]
				\centering
				\begin{tabular}{|c|c|c|c|c|}
					\hline

					№ & $T, K$ & $h, мм$ & $P, Па$ & $\sigma, мН/м$\\ \hline
					1 & $298.3 \pm 0.1$ & $188 \pm 1$ & $289 \pm 3$ & $78 \pm 4$\\ \hline
					2 & $300.4 \pm 0.1$ & $187 \pm 1$ & $288 \pm 3$ & $77 \pm 4$\\ \hline
					3 & $303.4 \pm 0.1$ & $186 \pm 1$ & $286 \pm 3$ & $77 \pm 4$\\ \hline
					4 & $306.4 \pm 0.1$ & $185 \pm 1$ & $285 \pm 3$ & $77 \pm 4$\\ \hline
					5 & $309.3 \pm 0.1$ & $184 \pm 1$ & $283 \pm 3$ & $76 \pm 4$\\ \hline
					6 & $312.3 \pm 0.1$ & $184 \pm 1$ & $283 \pm 3$ & $76 \pm 4$\\ \hline
					7 & $315.2 \pm 0.1$ & $183 \pm 1$ & $282 \pm 3$ & $76 \pm 4$\\ \hline
					8 & $318.2 \pm 0.1$ & $182 \pm 1$ & $280 \pm 3$ & $75 \pm 4$\\ \hline
					9 & $321.3 \pm 0.1$ & $181 \pm 1$ & $279 \pm 3$ & $75 \pm 4$\\ \hline

				\end{tabular}
				\caption{Результаты $\sigma(T)$}
				\label{tab:sigma_T}
			\end{table}

		\subsection{График зависимости $\sigma(T)$}

			\begin{figure}[!ht]
				\centering
				\includegraphics[width=0.75\textwidth]{sigma_T_plot.png}
				\caption{График зависимости $\sigma(T)$}
				\label{plot:dsigma_div_dT}
			\end{figure}



			\begin{align*}
				\frac{С_{P(П)} - C_{P(Ж)}}{L_0} \sigma_0 &= (-74 \pm 4)~\frac{мкН}{м*К}\\
				\left( \frac{d\sigma}{dT} \right) &= (-119 \pm 9)~\frac{мкН}{м*К}
			\end{align*}

	\section{Вывод}

		В ходе работы измерили температурную зависимость коэффициента поверхностного натяжения дистиллированной воды при различной температуре. А также определили полную поверхностную энергию и теплоту, необходимую для изотермического образования единицы поверхности жидкости.

		Несовпадение полученных результатов с табличными значениями можно объяснить загрязнённостью иглы и дистиллята.

\end{document}
